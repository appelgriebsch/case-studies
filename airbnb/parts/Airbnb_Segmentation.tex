%!TEX root = ../Airbnb_CaseStudy.tex
\section{Segmentation}
\label{sec:Segmentation}

This section describes the segmentation, that will be used to provide a usable database to analyse the collected information. The segmentation is divided into larger segments, that covers the three main interests for analysis: acquisition related information, behaviour on the website and the outcome of the users activity.

\subsection{Acquisition}
The first one is "Acquisition", which tries do identify general information about the visitor like demographic and technical parts. With this information we might for example interfere for which countries, genders etc. we need to improve marketing and also which technical settings might cause errors. But also important for the analysis are the origin of the visitor in regards to identifying if a marketing strategy worked (e.g. which search engine, which keywords etc.). In order to identify, if our website was interesting enough for people to revisit, we will also view at revisits.

\begin{itemize}
\item demography
\begin{itemize}
\item country
\item city
\item age
\item sex
\end{itemize}
\item origin of traffic
\begin{itemize}
\item email campaign
\item paid search
\item generic search (including search terms)
\item ads
\item social network
\end{itemize}
\item technical information
\begin{itemize}
\item operating systems,
\item browsers,
\item screen resolutions,
\item browser language,
\item mobile vs. desktop users
\end{itemize}
\item new / returning visitor /  former customers
\end{itemize}

\subsection{Behaviour}
The segment "Behaviour" covers the actual activities on the website except for some outcome related information. The segmentation of the verifying process and the used search options allow to identify features, that are highly used or rarely used. Based on this information it is possible to substitute particular features or improve them.

\begin{itemize}
\item verifying user profile
\begin{itemize}
\item Facebook
\item Google Plus
\item Linked In
\item e-mail
\item mobile phone
\item passport
\end{itemize}
\item find accommodation
\begin{itemize}
\item by neighbourhoods
\item by current location (mobile GPS service)
\item by regular onsite search (including search terms)
\item comparing places
\item communicate with host
\item make a reservation request
\end{itemize}
\item type of visitors
\begin{itemize}
\item hosts
\item customers
\item host and customer
\end{itemize}
\item list accommodations
\item promote accommodations
\begin{itemize}
\item on social networks
\begin{itemize}
\item Facebook
\item Twitter
\item Linked In
\item Google Plus
\end{itemize}
\item invite friends service
\item wishlists
\end{itemize}
\item payment preferences
\begin{itemize}
\item PayPal
\item credit card
\item bank transfer
\item direct deposit
\end{itemize}
\item managing booking
\begin{itemize}
\item get letter of approve
\item refund money
\item cancel
\end{itemize}
\end{itemize}

\subsection{Outcome}
In this section we measure  the artefact's of the customer's visit. What is the real value for the site we obtained after the visitor left? Measuring of the following parameters would be important from the KPI calculation point of view. The data must be connected to other gathered parameters via the session information.

We want to save this list of Actions:
\begin{itemize}
\item list a place
\item delist a place
\item rate a booking/accommodation
\item abort/cancel a booking
\item promote on social nework
\begin{itemize}
\item Facebook
\item Twitter
\item Linked In
\item Google Plus
\end{itemize}
\item complain about a booking
\item added a booking to wishlists
\item invite friends in service
\end{itemize}
Also we want to save this list of parameters:
\begin{itemize}
\item time to make a booking
\item amount of clicks to make a booking
\item conversion rate of promotion
\end{itemize}
