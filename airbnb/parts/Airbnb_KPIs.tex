%!TEX root = ../Airbnb_CaseStudy.tex
\section{Objectives and KPIs} % (fold)
\label{sec:obj_kpis}

To track the progress toward the operational and strategical objectives of a company it has to define and monitor {\bf quantifiable} measures. Those measures have to be able to show the success or failure of certain targets. There might be some common industry key performance indicators (KPI) but nevertheless to get most out of the KPIs you use to monitor your web site's success they have to be synchronised with the company's strategy \citep[p. 347-349]{Kaushik07}.

\subsection{Establish a trusted p2p marketplace}
To figure out if the users still have trust in the p2p marketplace we need to collect information about the number of customer complaints, how many of them could not be solved directly as well as the \textit{transaction throughput} on the system. This will give us also hints about the \textit{customer satisfaction} and allows us to see a decline in the usage patterns of the platform more rapidly. In addition we can get feedback from our customers about our \textit{problem solving strategies} in case they are facing issues with the host.

\begin{description}
	\item \textbf{Possible KPIs}
	\begin{itemize}
    	\item No. of returning customers 2+
        \item Quality survey about "have you at some point felt like you won't be receiving your money" or "have you at some point felt like your host won't be showing up?"
        \item Rating of host in survey
        \item Ratio of mitigated to unmitigated conflict cases
        \item No. of transactions processed per period
        \item Amount of transactions processed per period
        \item No. of revisits per period
    \end{itemize}
\end{description}

% subsection p2p_marketplace (end)

\subsection{Available to and used by a worldwide community}
As one of the main objectives of our service is to establish a platform that is used on a global scope we have to measure the number of customers per region, the number of customer registration per region as well as a ratio of places available to book in various cities around the world. This will provide us with information if we succeed to \textit{attract new users} and \textit{to enlist additional outstanding places}. As our platform relies on a \textit{network effect} to continue to grow we have to continuously attract new users who might recommend the system to their friends or might also offer some interesting places to stay.

\begin{description}
	\item \textbf{Possible KPIs}
	\begin{itemize}
    	\item No. of cities with more than 20\% places to book
        \item No. of new customers from one country
        \item No. of countries with more than N customers
        \item Track origin of requester
        \item Ratio of visitors to population of the country
        \item No. of visitors from particular country
    \end{itemize}
\end{description}

% subsection worldwide_community (end)

\subsection{Enlist unique, attractive accommodations}
Just having a large pool of accommodations in place might not be sufficient to keep the offers on the platform attractive. We will have to constantly measure the actual set of offerings to figure out if they still \textit{fit the client expectations}, if the \textit{quality of the service} at the location is either steadying or growing or whether we \textit{missed out some regions} in a country that are of interest to our user base.

\begin{description}
	\item \textbf{Possible KPIs}
	\begin{itemize}
    	\item Location of accommodations (GPS coordinates to figure out distribution over a region)
        \item Ratio of special ``stop words" that are used in descriptions/reviews
        \item Rating of accommodations in customer survey (satisfaction, uniqueness)
        \item No. of sharing of accommodations on social networks (recommendation of place through the customer)
        \item No. of likes per accommodation on our own platform
    \end{itemize}
\end{description}

% subsection unique_accomodations (end)

\subsection{Provide a convenient booking experience}
The \textit{ease-of-use of our platform} is important to speed-up the process of finding and booking a desired place for the customer. So we measure how long it takes for a customer to finish the process, how long they stay on each site, which steps they take on the site to reach their goal as well as the number of aborts and cancellations, whose can give us hints if the process or the web site itself might be to complicated.

\begin{description}
	\item \textbf{Possible KPIs}
	\begin{itemize}
    	\item No. of unsatisfied customers (complaints \& refunds) per 1000 bookings
        \item Increase conversion ratio of visitors who booked a place
        \item Increase conversion ratio of visitors who became hosts
        \item No. of booking problems (aborts, cancellations, bouncers, ...)
        \item Major conversion rate
        \item Minor steps taken to book a place
        \item Time it took from visiting the site till the booking
        \item Time spend on each page
    \end{itemize}
\end{description}

% subsection convenient_experience (end)

\subsection{Leading to loyal, satisfied customers}
From a technical point of view we want to make sure that our platform is \textit{always available} for the customer and there are \textit{zero down-times} due to server or software errors. Additionally we want to make sure that the system \textit{does not slow down} with an increasing number of concurrent users accessing it. Beside those technical issues we also want to make sure that the existing customers are well served and get help immediately if it is required.

\begin{description}
	\item \textbf{Possible KPIs}
	\begin{itemize}
    	\item No. of concurrent users
    	\item Network latency
        \item Page load time (due to amount of processes)
        \item No. of 404 pages
        \item No. of 5xx pages (server errors)
        \item Server down-time/up-time
        \item No. of users reacting on mail campaigns
        \item No. of users utilizing the help system
    \end{itemize}
\end{description}
% section obj_kpis (end)
