\todo{Perhaps delete social movements section - or rename to social change and include how social movements fit into social change?}
\subsection{Social Movements}
\label{sec:socmovement}
The sociologist and author of the book \textit{Social Movements 1768-2004} proposes three things aspects of Social Movements:
\begin{enumerate}
\item They develop public campaigns, getting organised to make collective claims on targeted audiences.
\item They combine whole repertoires of political actions, ranging from public meetings, processions and rallies through demonstrations, petitions and the creation of special purpose associations.
\item And ultimately they display and present themselves to the public as good causes and worthy people. They are united, with large numbers of committed supporters.
\end{enumerate}
Change.org and similar on-line petition sites play a key support role in this process. Traditionally, petitions seemed to have been a way to show that an issue has a substantial amount of support when it reached some threshold of number of signatures. However, the modern adaptation - epetitions - can be shared easily \todo{WC: remove later: modified easily?} and otherwise promoted across various information distribution channels available on the Web. This way, not only do they show support like the traditional petition, they all also promote the issue with the potential to attract additional sup
porters.
\todo{WC: review Social Movements section}

\section{Business Case}
\label{subsec:businessCase}
We found the current description of \textbf{digital democracy} in the textbook to be insufficient in detail \citep[p. 815]{Plummer11}, and the way in which the tools of digital democracies were driving political action, especially with regard to environmental issues, to be understated. This concept deserves a more in-depth analysis. A better understanding of digital democracies will help anyone from all attempting to create change around a real world issue, have their voice heard on, potentially, a global stage. \par \vspace{0.2cm}
\textbf{On-line petitions} are an aspect of the digital world that individuals, businesses and governments can both benefit from or be afflicted by. Looking at Change.org in particular it is seen that on-line-petitions are proving to be effective and accomplishing change at each level -- locally, nationally and globally.\par \vspace{0.2cm}
Social action can be effective at a very personal as well as on wider organisational level. Large businesses with the knowledge that suppliers, customers, users, employees and other stakeholders have such a system at their disposal should consider operating in a manner that keeps their stakeholders happy and less likely to create negative publicity. Consumers have a powerful vehicle through Change.org to improve the products and services they use. Acknowledging what stakeholders want and responding favourable can create benefit to the image of company in the eyes of the public.\par \vspace{0.2cm}
Change.org has also had numerous success changing government policy in response to the demands of the community in countries around the world. Several examples include President Obama signing an extension of the visa program for Iraqi interpreters \citep{Change15}, the South African government agreeing to launch a National Task Team to end 'corrective' rape \citep{Change13} and the Indian Government agreeing to regulate and control the sale of acid \citep{Change12}.
\todo{WC: check if we should come up with the samples here}
\subsection{Unique Features of Our Project}
\label{sec:uniquefeatures}
In summary, from the sample presented here and other literature, we have found works that addressed similar topics to that which we will endeavour to. However, although each paper strongly focussed on a particular aspect of driving political change, for example -- the impact of blogs, the impact of social networks, we were unsatisfied with what we found for the impact of on-line petitions as a political driver. As well as this, the literature focusing particularly on environmental action in this area seemed to be very elusive for us. Therefore, we believe that we have found a niche for which the unique features of our project could fulfil. \par \vspace{0.2cm}
There are few theoretical frameworks offered to us from Sociology which help us to understand the problem that Change.org and other tools of digital democracy are addressing, especially in regards to environmental protection:

\begin{itemize}
  \item Sociology analyses Democracies as a type of political system and the types problems (p541) that is facing today. This includes the seemingly outdate method of voting once every few years a party to represent oneself in the political forum when outside the political forum such as online petitions and twitter are able to provide public feedback on issues almost in real time.
  \item Globalisation and politcs (p545). We look at how members all around the world can vote on issues that put pressure on local governments.
  \item 'Gender and power' is looked at p546, it is noted that 18.4 \% of the world's parliaments are women, do online petition offer a more equal participation rate for females to reflect power. A key theme behind change.org is that anyone can start a petition.
  \item The pluralist model of power p550, views power as dispersed among many competiting interest groups, similarly to how change.org disperses power, in support of online petitions, however, do the tools of digital democracy challenge the power elite model?
\end{itemize}


% subsection Unique Features of Our Project (end)