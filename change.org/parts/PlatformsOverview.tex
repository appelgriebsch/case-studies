\subsection{Primary case overview: Change.org}
\label{sec:PlatformsOverview}
% The following paragraph introduces our case study
We have chosen to do descriptive case studies. The Big Idea section articulates how digital democracies are driving social change for environmental protection in theory. Here we intend to mine real-world cases from Change.org and other on-line petition sites for 'abstract interpretations of data and theory development' \citep{casestudyResearch}.

\subsubsection{Brief history}
The site Change.org was launched on February 7, 2007. In 2008, Change.org cooperated with MySpace to ask the activists to submit policy ideas to the Obama administration \citep{Stirland2008}. During 2011, Change.org grew from 20 employees to 100 employees and offices on four continents\citep{Kristof2012}. Change.org has grown from six million users in the beginning of 2012 to more than 35 million users in May of 2013, and has become one of the largest and fastest growing petition platforms \citep{Empson2013}. In May 2013 the company has raised \$15 million investment from Omidyar Network \citep{Pozin2013}. In June 2014 the company announced a reach of more than 70 million users in 196 countries \citep{Change14a}. 
\subsubsection{Business model}
Change.org's business model is based on sponsored petitions. It is possible for the petition starter to pay for promotion of the petition among registered users. Most of the advertisers are non-profit organizations. Change.org claims that users will receive only those sponsored petitions the user interested in. Change.org pays close attention to the topics of petitions. They do not allow any hateful or discriminatory content. Another key point is that they never share private information without permission \citep{Change16}.\par\vspace{0.2cm}
One criticism of Change.org from users is that company is not clearly recognizable as for-profit organization.
\begin{quote}
The service is free, and with a name like Change.org the company even sounds like a not-for-profit. But it’s not. It was founded in 2007 and spent the better part of two years flailing around for a profitable business model until Rattray hit upon a clever approach. Change.org charges groups for the privilege of sponsoring petitions that are matched to users who have similar interests. For example, when a person signs a petition about education and clicks “submit,” a box pops up and shows five sponsored petitions on education to also sign. If a user leaves a box checked that says “Keep me updated on this campaign and others,” the sponsor can then send e-mails directly to that person. It’s not clear from the check box that your e-mail address is being sold to a not-for-profit. Rattray says an imminent site redesign will make the company’s business model more transparent. Change.org has 300 paying clients, including Sierra Club, Credo Wireless and Amnesty International, and its revenue so far this year is \$15 million.
\citep{Geron2012}
\end{quote}

\subsubsection{Victorious environmental cases}
Change.org claims over 6,000 victories and 50 million users. We investigated the site to find noteworthy cases of successful online petitions in support of the environment. The list of the cases can be found in \ref{subsec:VictoriousCases}. The minimum number of signers for considered cases is 259 and the maximum is 290,342. 

\subsection{Alternative platforms overview}
\subsubsection*{Avaaz.org}
Avaaz is an online campaigning organisation that is halfway between an NGO and a megaphone. Avaaz means "voice" or "song" in Persian \citep{Butler2013}.
Avaaz' mission is to close the gap between the world we have and the world most people everywhere want. Their community is unique in its ability to mobilize citizen pressure on governments everywhere to act on crises \citep{Avaaz2011}.\par\vspace{0.2cm}

The platform has 35 million members in 194 countries. Avaaz is \textbf{wholly member-funded}. They have no corporate sponsor or government backer, which may insist on an external agenda. They do not accept funds from governments or corporations.
\begin{quote}
Avaaz is both global and globalised and its approach is less bleeding-heart liberal than hard-headed pragmatist. It doesn't launch a campaign – to save fin whales from being butchered, or to free trapped migrant workers in Bahrain, or to bring peace to Palestine – because Patel, or the staff passionately believe in it (though they might). They launch a campaign that they think will fly. It's tried out on a sample of members and then they gauge the response. Patel says it's a way of ensuring that the members have the ultimate power, that they're the boss, not him.
\citep{Cadwalladr2013}
\end{quote}

\subsubsection*{Care2.com}
Care2 is a trusted social action network that empowers millions of people to lead a healthy, sustainable lifestyle and support socially responsible causes. Its content offering includes original stories, blogs and syndication partners covering a wide range of healthy lifestyle areas, and causes ranging from politics to human rights and animal welfare. By integrating relevant content with action opportunities such as petitions, pledges and daily actions \citep{Care22014}. Care2 owns and operates the site for petitions, \url{www.thepetitionsite.com}.\par\vspace{0.2cm}

Care2 is a profitable, privately funded company and a B-Corporation. The company's business model is focused on pay-per-action lead generation for non-profit organizations and CPM sponsorships for responsible consumer brands. As a B-Corporation or social enterprise, Care2 generates revenues by connecting individuals with nonprofits and businesses.\citep{Care22014} They anounced 25,9 Million members predominantly from US.

\subsubsection*{38degrees.org.uk}
38 Degrees is one of the UK's biggest campaigning communities, with over 2.5 million members. The organisation is "people-powered". They don't take money from political parties, government or big business, but \textbf{rely on donations from members}. 38 Degrees ask members to donate to support work or to fund a specific campaign or a specific action (e.g. to pay for the costs of organising a demonstration or putting up adverts). 38 Degrees has also received some money from charitable trusts and foundations. \citep{38degrees2014}
\par\vspace{0.2cm}
\begin{quote}
The campaigns that members are asked to support within 38 Degrees are simple one-step actions – signing a petition or sending a message to an MP – rather than more traditional, complex long-term campaigns.
This might mean that members take more than one discrete action towards achieving a positive outcome on the issue at hand, but the actions themselves are always meant to be simple, tangible steps toward that result. \citep{Puckett2011}
\end{quote}
As another example of activities, 38 Degrees organised debates with participation of UK parties to bring an in-depth look at the issues their members wanted to raise with the authors of each party's manifesto. \citep{Guardian2010}

\subsubsection*{Petitions.MoveOn.org}
In 2012, the liberal group MoveOn.org, one of the early successes of the digital political age, shifted to make member-driven petitions the center of its organizing efforts.\citep{Weiner2013}
\begin{quote}
Petitions can be started by any person or organization, not just MoveOn members. If a petition gets 20 signatures, MoveOn staff will send it to a small test group of members. If that group is enthusiastic, the petition will be sent to a larger group. If it continues to pick up steam, MoveOn staffers will work with the petitioners on running a campaign, including media training, phone calls, event organizing and fundraising help.
\citep{Weiner2013}
\end{quote}
MoveOn is entirely funded by small \textbf{donations from its members}. It consist of more than 8 million Americans.
\par\vspace{0.2cm}
Also, progressive state legislators and organizations such as Progress Now, Ultraviolet, Faithful America, the AFL-CIO, Free Speech for People, Social Security Works, Working Families Party, Color of Change, and the Organic Consumers Association have used MoveOn's petition website to give their campaigns a major boost.

\subsubsection*{Getup.org.au}
GetUp is an independent, grass-roots community advocacy organisation which aims to build a more progressive Australia by giving everyday Australians the opportunity to get involved and hold politicians accountable on important issues\citep{getup2014a}.
\par\vspace{0.2cm}
GetUp was founded in 2005 by a team of three Australians - Jeremy Heimans, David Madden and Amanda Tattersall. David and Jeremy are Australian graduates of Harvard University's Kennedy School of Government who have worked at the intersection of technology, new media and politics in the United States. David and Jeremy also co-founded Avaaz.org, a global online political community inspired by the success of GetUp and the US group MoveOn.org. Amanda Tattersall is a Sydney based community organiser who has founded a series of new social change organisations in Sydney and is known for her writing on civil society collaboration\citep{getup2014b}. 
\par\vspace{0.2cm}
GetUp is a \textbf{not-profit organisation} and relies on small donations to fund its work and in-kind donations from the Australian public. GetUp does not accept donations from political parties or the Government. It has over 630,000 members. In Anual report for 12/13 Financial Year the company reported \$ 4 061 664 donations \citep{getup2013}

\subsubsection*{Others}
Govermental organisations also have their own sites for petitions. The list of the most known is:
\begin{description}
\item[petitions.whitehouse.gov] "We the People". A section of the US White House website, launched in 2011.
\item[epetitionen.bundestag.de] A section of the Germany's Bundestag website, launched in 2005.
\item[epetitions.direct.gov.uk] A website for petitioning government and Parliament in the UK, launched in 2011.
\end{description}