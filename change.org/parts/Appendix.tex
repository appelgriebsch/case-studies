\appendix
\section{Assumptions}
\label{sec:assumptions}

In this section we offer an examination of our assumptions going into this project. The examination is based on Susan Drays \textit{Questioning Assumptions: UX Research That Really Matters} \citep{dray2014questioning}. We have based our discussions for this section especially on \textbf{user typologies} and \textbf{value proposition stories}. Dray also suggests examining the \textbf{Design concept and direction}, however as we are not familiar with writing sociology papers, so we have not been able to observe, this stage, any 'default' concept decisions.

\subsection{Assumptions about Change.org}
\label{subsec:keyrequirements}
In our reflection of our assumptions, we have realised that had some assumptions about how Change.org works. We have already begun to address these assumptions and replaced them with information supportable with evidence. Our initial assumptions about Change.org included:
\begin{enumerate}
	
    \item people will use Change.org to speak out on environmental protection issues
    \item they will always know whom to address with their complaint
    \item they will keep driving the issue until the end and will not loose interest in-between
    \item they will take anything that’s needed to bring the issue to a positive social change (even if that means they have to spend a lot of personal time and maybe also money to support the case)
    \item they are highly connected in the on-line world and can easily promote their issue
    \item there is free access to the internet available (without any restrictions or state-controlled press)
    \item the person in charge for handling the case on governmental site has the power to make valid decisions
    
\end{enumerate}

\subsection{The Basis of Those Beliefs}
\label{subsec:basisofbeliefs}

The basis for those beliefs is in:

\begin{enumerate}

	\item we need to use the network effect to spread the case rapidly
    \item the issuer of the case have to maintain the discussion on the web site, answering questions from may-be supporters, stay in contact with person in charge or the press to keep the issue alive 
    \item developed countries choose the idea of preserving resources over the focusing on making money

\end{enumerate}
\subsection{Quality of the Evidence}
\label{subsec:qualityofevidence}

In finding a need for our project, there was an adequate amount of evidence suggesting that it could be a useful resource. We found, as shown in the related works, that there is literature approaching similar drivers of political change arising from web technologies such as blogs and social networks. However, the literature was lacking on the connection between on-line petitions and political action. \par
\vspace{0.2cm}
On the other hand, as there is a limited about of literature for on-line petitions regarding political action, the evidence supporting some of our assumptions may not be available or of high quality. For example, we may have to solely reply on the biasedly presented Change.org stories and light covering from local media to find evidence supporting some of our assumptions.

\subsection{Impact on the Project's Direction}
\label{subsec:systemsdirection}

A possible conclusion of the project might be that doing on-line petition might work for smaller, more local issues, but might fail when dealing with such a global approach that is required to deal with the environmental protection case.

\subsection{The Risk That These Assumptions are Wrong}
\label{subsec:riskofassumptions}

Those assumptions made in the previous section are plausible expectations for an on-line community-driven social project. They resembles the points we see important to make a positive social change happen in a virtual environment.

\subsection{Plausible Alternatives}
\label{subsec:plausiblealternatives}

If this will not work out at the end there is still the possibility to protest against misbehaviour of the government or economical leaders on the street the old way.

\subsection{Possible detailed plan for the interiew}
\label{subsec:interviewPlan}
\textbf{ Richard W. Halsey Interview Plan}
\begin{itemize}
\item \textbf{1 July 11:00am-12:00pm:} Iaroslav interviews Richard W. Halsey and records it using a sound recorder, the director of The California Chaparral Institute. The California Chaparral Institute petitioned the California Board of Forestry to "Stop Plan to Target 38 Million Acres of Habitat in California for Clearance" on change.org: \url{http://www.change.org/petitions/california-board-of-forestry-stop-plan-to-target-38-million-acres-of-habitat-in-california-for-clearance}
\begin{enumerate}[label*=\arabic*.]
\item
The semi-structured interview begins with the questions:
\begin{itemize}
\item
What is your activism experience? The rationale for this question is: breaks the ice with the interviewee, they only have to talk about their own experiences. Also provides insight into the type of environmental context they are working from.
\item What were the motives of creating the petition?
Rationale for question: Explore the motives of the starter. Explore the possible items for the expectation list.
\item Was the publication of the petition a primary action in your campaign?
Rationale for question: Explore the context of the case. 
\item What were your expectations from the petition?
Rationale for question: Draw out aspects of online petitions which the subject believes make online petitions effective. Explore possible expectations from petitions.
\item What other online promotional actions were done?
Rationale for question: Explore the context. Identify other aspects which could influence the decision.
\item What offline promotional actions were done?
Rationale for question: Explore the context. Identify other aspects which could influence the decision.
\item What medias did publications about your campaign?
Rationale for question: Explore the context. Identify other aspects which could influence the decision.
\item What was, from your point of view, the main reason for the decision-maker to accept your petition?
Rationale for question: measure the actual response compared to expectations of what the petition would bring. Explore the items for reasons list.
\item What can be other reasons for the decision-maker to accept a petition?
Rationale for question: Explore the items for reasons list.
\end{itemize}
\item
The following questions will be made up in order to explore particular aspects in more detail. 
\end{enumerate}
\item  \textbf{1 July 12:00pm-1:00pm:} Iaroslav summarises key points. The key points and audio recording of the interview are email to William and Andreas.
\item \textbf{1 July 3:00pm-5:00pm:} Andreas writes up a transcript of the interview and attaches it to the appendix of the project.
\item  \textbf{4 July 2:00-2:30pm:} Iaroslav, Andreas and William conduct a video conference to collectively deduce key points about why online petitions are or are not effective based on the Halsey interview. William summarises key findings at the end of the meeting. Key findings should primarily answer at least the first four questions from the interview, and other notable results regarding the effectiveness or ineffectiveness of online petitions should be included.
\item  \textbf{4 July 3:00pm-5:00pm:} William includes key findings in the 'Key Findings from Interviews' section in the Project document.
\end{itemize}


The direction may be changed according to show other on-line methods and tolls for solving environment protection problems. 

% section assumptions (end)

\subsection{Victorious environmental cases on Change.org}
\label{subsec:VictoriousCases}
In this section we offer a shortlist of successful online petitions in the area of environmental protection from Change.org. This list has been collated in order to get a better understanding of what types of environmental protection issues could be represented by online petitions as well as provide a starting point for the investigation into whether the environmental protection efforts represented by each of these cases was really driven by the online petitions themselves or some other, external factors. Each online petition is given with its title, URL, number of signatures and the final verdict as stated on the Change.org site.

\begin{itemize}
\item Yukon Government: Extend the ban. Accept the plan.
\par\vspace{0.1cm}
\url{http://www.change.org/en-GB/petitions/yukon-government-extend-the-ban-accept-the-plan}
\par\vspace{0.1cm}
Number of signatures:	5,128
\par\vspace{0.1cm}
Verdict: Yukon Government announced the extension of the staking ban on the Peel Watershed

\item Tell Amazon.com To Permanently Ban The Sale Of Whale, Dolphin, and Porpoise Meat 
\url{http://www.change.org/en-GB/petitions/tell-amazon-com-to-permanently-ban-the-sale-of-whale-dolphin-and-porpoise-meat}
\\Number of signatures: 202,961
\\Verdict: Amazon.com officially put a policy in place against selling whale and dolphin meat after quietly pulling the offending products and refusing to issue a permanent ban. Amazon’s updated list of prohibited items now includes a ban on “parts or products from whale or dolphin.”

\item Rajasthan State Pollution Control Board: Shut down Asphalt factories in Karauli (Rajasthan)
\url{http://www.change.org/en-IN/petitions/rajasthan-state-pollution-control-board-shut-down-asphalt-factories-in-karauli-rajasthan}
\\Number of signatures: 3,616
\\Verdict: Rajasthan State Pollution Control Board found that most of these factories were illegal and were crossing their pollution limit. A month later the first factory was shut down in village Asthal (Karauli). On 27th November ' 2012 a Senior Environment engineer informed that all the 5 factories were sealed. He also mentioned that they will remain sealed unless they get valid licences and comply with the norms.

\item Plant 4000 trees along Hubballi-Dharwad BRTS Highway
\url{http://www.change.org/en-IN/petitions/plant-4000trees-along-hubballi-dharwad-brts-highway}
\\Number of signatures: 20,099
\\Verdict: The Hubli Dharward BRTS Co. Ltd has committed to plant not just 4000 trees as avenue plantation but 18,000 trees to compensate for the 1800 trees that were cut.

\item Iowa Utilities Board: NO NUCLEAR PLANT IN MUSCATINE COUNTY
\url{http://www.change.org/petitions/iowa-utilities-board-no-nuclear-plant-in-muscatine-county}
\\Number of signatures: 259
\\Verdict: MidAmerican decides against Nuclear Power Plant in Iowa. They are opting to expand Wind Power instead.

\item Order Otter Tail Power to replace its Hoot Lake Plant with clean energy.
\url{http://www.change.org/petitions/minnesota-public-utilities-commission-order-otter-tail-power-to-replace-its-hoot-lake-plant-with-clean-energy}
\\Number of signatures: 474
\\Verdict: The Minnesota Public Utilities Commission revoked the Hoot Lake Coal plants permit to burn toxic coal. The plant will stop burning coal in 2020.

\item Santa Barbara County Supervisors and State Representatives: Stop New Dirty Oil Development
\url{http://www.change.org/petitions/santa-barbara-county-supervisors-and-state-representatives-stop-new-dirty-oil-development}
\\Number of signatures: 2,581
\\Verdict: Santa Barbara County Supervisors narrowly voted to require a massive, energy-intensive oil project (the Santa Maria Energy project) to offset the majority of their greenhouse gas emissions. They cited the large number of petitions and letters they received opposing the project as a factor. This single decision will result in offsets of 40,000+ of tons of greenhouse gas pollution per year.

\item Keep Oil Exploration Out of Virunga National Park
\url{http://www.change.org/petitions/keep-oil-exploration-out-of-virunga-national-park-2}
\\Number of signatures: 290,342 
\\Verdict: Soco International PLC, the London-based multinational oil company, announced that it would end controversial oil exploration inside Virunga. Moreover, Soco has committed to stay out of all other World Heritage Sites around the globe. 

%\item 
%\url{}
%Number of signatures:
%Verdict:
\end{itemize}

\section{Risks}
\label{sec:risks}
Here we identify risks that we are likely to face while completing our project. Risks are apparent both in the content we are writing about as well external factors which impact our ability to deliver our project. Our risks are listed here, and elaborated below:\par
\vspace{0.2cm}

\begin{itemize}

  \item Legitimacy and information-availability of the sources used for examples
  \item Time scheduling issues
  \item Misunderstanding formal requirements

\end{itemize}

Our highest priority is providing examples of digital democracy in action; there are some risks that may afflict our efforts in this area. Firstly, given the nature of some of the petitions, especially those out of the spotlight of the media, it is possible that we will not be able to authenticate some stories. Secondly, as we have chosen to focus on the on-line petition tool, there is inherently the risk that although the petition reaches its target number of supporters, no action is taken at the end; or the action is not reported on. That is, there may not be a direct causal link between the petition and the resultant political behaviour. Finally, as Change.org is a for-profit business, it is likely bias in its presentation to show itself as an effective tool. For example, internal Change.org staff pick the featured stories \citep{ForbesChange}. Given that our highest priority is to provide examples of digital democracy in action we will need to carefully consider the legitimacy and context of the examples we use as well also look at secondary sources to cross check the available information.\par
\vspace{0.2cm}
The other two risks involve external factors that may afflict our ability to work effectively on the project. We have an interesting group dynamic, one member is in Russia and another must work within some of business hours of the Australian time-zone. Thus far, this had made scheduling meetings problematic as they must take three time-zones into consideration. Indirectly, this has also affected our ability to effectively communicate our ideas to one another. Since the beginning of the course we have found helpful collaboration tools such as WriteLatex.com and Google Hangouts which have helped mitigate this risk.\par
\vspace{0.2cm}
The second external factor is the potential for misunderstanding the formal requirements. We are all new students of sociology. The majority of our time was absorbed in discussions over exactly what our project should be about - and whether or not different topics were relevant to the study of sociology. If there was no time limit on the project, this discussion would be continuing. Although, importantly, We have all made an effort to read relevant sections of the sociology book and are unanimously satisfied with the topic we have come up with. However, the risk remains that our limited understanding of what goes into a sociology paper may have us overlook certain elements which will impact our final project.\par
\vspace{0.2cm}
Having identified risks to the project associated intrinsic properties of on-line petitions and Change.org itself, as well as external factors; we are in a position to better mitigate these risks.

% section risks (end)

\section{Project Plan}
\label{sec:proj_plan}
\ganttset{%
    calendar week text={%
    	CW\currentweek
	}%
}
\begin{figure}[H]
\centering
\resizebox{0.91\textwidth}{!}{
\begin{tikzpicture}
  \begin{ganttchart}[
  	  today=2014-06-29,
      vgrid={*6{draw=none}, dotted},
      hgrid,
      time slot format=isodate,
      x unit=.16cm,
      y unit chart=1.3cm,
      inline,
      milestone inline label node/.append style={left=5mm}
  ]{2014-03-31}{2014-06-29}
  
      \gantttitlecalendar{year, week=14} \\
      \ganttgroup{Web \& Society Project}{2014-04-01}{2014-06-29} \\
     
      \ganttbar{Define problem space}{2014-04-01}{2014-05-08} \\
      \ganttbar{Write Exposé}{2014-04-20}{2014-05-08} \\
      
      \ganttlink[link type=f-f]{elem1}{elem2}
      \ganttlinkedmilestone{Exposé}{2014-05-08} \\
    
      \ganttbar{Choose reviewers}{2014-05-12}{2014-05-18} \\
      \ganttbar{Detailed problem definition}{2014-05-08}{2014-05-27} \\
      \ganttbar{Evaluate related work}{2014-05-08}{2014-05-27} \\
    
      \ganttlink[link type=f-f]{elem5}{elem6}
      \ganttlinkedmilestone{Concept}{2014-05-27} \\
      
      
      \ganttbar{Write down Environment as sociology topic}{2014-05-30}{2014-06-10} \\
      \ganttbar{Write down Social Movements}{2014-05-30}{2014-06-10} \\
      \ganttbar{Write down New Technologies and Risk society}{2014-05-30}{2014-06-14} \\
      
      \ganttmilestone{Draft \#1}{2014-06-05} \\
      
      \ganttbar{Check success stories from Change.org}{2014-06-06}{2014-06-14} \\
      \ganttbar{Check how Change.org drives social movements} {2014-06-06}{2014-06-14} \\
      
      \ganttmilestone{Draft \#2}{2014-06-10} \\
      
      \ganttbar{Analyse success factors on Change.org}{2014-06-11}{2014-06-20} \\ 
     
      \ganttbar{Analyse how social change can be achieved by on-line petition}{2014-06-11}{2014-06-20} \\
     
      \ganttbar{Analyse alternatives to Change.org}{2014-06-06}{2014-06-20} \\
      
      \ganttbar{Finalize document}{2014-06-20}{2014-06-28} \\
      \ganttbar{Create Poster}{2014-06-22}{2014-06-28} \\
      
      \ganttmilestone{Final}{2014-06-29}
  
  \end{ganttchart}
\end{tikzpicture} 
}
\caption{Project Plan}
\end{figure}

\begin{table}[H]
\renewcommand{\arraystretch}{1.6}
\begin{threeparttable}

  \caption{Time durations of phase: Defining a problem (hours)}
  
  \begin{tabular}{p{3cm} c|c c c| c}
      \textbf{Task} & \textbf{Date}& \textbf{Clear} & \textbf{Gerlach} & \textbf{Popov} & \textbf{Total hours} \\
%					- 	& date		 & Will & Andr.	& Iar. 	& Total \\ 
 		On-site session
        				& 15.03.2014 & 1.5 	& 1.5 	& 1.5 	& 4.5 \\
	   	On-line meeting	& 01.04.2014 & 3 	& 3 	& 3 	& 9 \\
       	Write options 	& 01.04.2014 & 1 	& 1 	& 1 	& 3 \\
       	Reconcile the problem
       					& 01.04.2014 & 1 	& 1 	& 1 	& 3 \\
%template row		- 	& xx.04.2014 & x 	& x 	& x 	& x \\ 

  \end{tabular}

\end{threeparttable}

\end{table}

\begin{table}[H]
\renewcommand{\arraystretch}{1.6}
\begin{threeparttable}

  \caption{Time durations of phase: Expos\'{e} (hours)}
  
  \begin{tabular}{p{3cm} c|c c c| c}
      \textbf{Task} & \textbf{Date}& \textbf{Clear} & \textbf{Gerlach} & \textbf{Popov} & \textbf{Total hours} \\
%					- 	& date		 & Will & Andr.	& Iar. 	& Total \\ 
		Choose format of the document
        				& 29.04.2014 & 1 	& 1 	& 	 	& 2  \\                 	  Check for related works on internet
                        & 04.05.2014 & 		& 2		&		& 2  \\
        Write formal sections 
                        & 04.05.2014 & 2 	& 2 	&  		& 4 \\ 
        Identify objectives
        				& 06.05.2014 & 1 	& 2 	& 1 	& 4 \\ 
        Define risks
        				& 06.05.2014 & 1 	& 1 	& 1 	& 3 \\ 
        Identify aspects \& perspectives
        				& 06.05.2014 &  	& 2 	&   	& 2 \\ 
        Write expos\'{e}& 07.05.2014 & 6	& 2		&  		& 8\\ 
        Write project plan
        				& 07.05.2014 &  	&  		& 4 	& 4 \\ 
        Identify assumptions
        				& 08.05.2014 & 2	& 2		& 2		& 6 \\
        Write expos\'{e}& 08.05.2014 & 5	& 2		& 5 	& 12  \\ 
        
        Present expos\'{e}
        				& 15.05.2014 & 3	& 3		& 3		& 9 \\
%template row		- 	& xx.04.2014 & x 	& x 	& x 	& x \\ 

  \end{tabular}

\end{threeparttable}

\end{table}

\begin{table}[H]
\renewcommand{\arraystretch}{1.6}
\begin{threeparttable}

  \caption{Time durations of phase: Concept (hours)}
  
  \begin{tabular}{p{3cm} c|c c c| c}
      \textbf{Task} & \textbf{Date}& \textbf{Clear} & \textbf{Gerlach} & \textbf{Popov} & \textbf{Total hours} \\
%					- 	& date		 & Will & Andr.	& Iar. 	& Total \\ 
        Check for related works
        				& 17.05.2014 &	2	&  4		&		& 4 \\
        Read related works
                        & 21.05.2014 &	1	&  		&		&  \\
        				& 22.05.2014 &	2	&  4		&		& 6 \\
                        & 24.05.2014 &	3	&  6		&		& 9 \\
                        & 25.05.2014 &	2	&  		&		&  \\
        Group Conf-Call
        				& 25.05.2014 &	2	&  2		&		& 4 \\
        Update expose to concept
        				& 25.05.2014 &  1	&  		&   	&  \\
                        & 26.05.2014 &  1	&  		&   	&  \\
        Write project plan
        				& 25.05.2014 &  	&  6		&   	& 6 \\
                        & 27.05.2014 &		&  3		&		& 3 \\
		Check for related works
        				& 29.05.2014 &		&  		&	6	& 6 \\                        
		Read related works                        
        				& 30.05.2014 &		&  		&	4	& 4 \\                        
						& 31.05.2014 &		&  		&	2	& 2 \\
                        & 01.06.2014 &		&  		&	3	& 3 \\
%template row		- 	& xx.04.2014 & x 	& x 	& x 	& x \\ 

  \end{tabular}

\end{threeparttable}

\end{table}

\begin{table}[H]
\renewcommand{\arraystretch}{1.6}
\begin{threeparttable}

  \caption{Time durations of phase: Draft \#1 \& \#2}
  
  \begin{tabular}{p{3cm} c|c c c| c}
      \textbf{Task} & \textbf{Date}& \textbf{Clear} & \textbf{Gerlach} & \textbf{Popov} & \textbf{Total hours} \\
%					- 	& date		 & Will & Andr.	& Iar. 	& Total \\ 
        \#1 feedback session
        				& 03.06.2014 &	1	&  1		&	1	& 3 \\
        Discussing next steps
        				& 03.06.2014 &	2	&  2		& 	2	& 6 \\
        Make appendicies\'{e}
        				& 04.06.2014 & 1	& 			& 		& 1 \\
        Research + Write Social Movements  
        				& 04.06.2014 & 3	& 			& 		& 3 \\                         Research + Write Environment as sociology topic
                        & 05.06.2014 & 		&  1		&		& 1 \\	
        Update project plan
                        & 05.06.2014 &		&  1		&		& 1 \\
        Research + Write platforms overview  
        				& 08.06.2014 & 	& 			& 	6	& 6 \\                        
        Restructure document
        				& 08.06.2014 & 1	&	2	   &		& 3 \\
        Write about environmental concepts
        				& 09.06.2014  &	&	3		&		&  3 \\
        Editing
        				& 09.06.2014  & 3	&			&		&  3 \\
                        & 10.06.2014  &		 &	2		&		&  2 \\
        Planning restructure
        				& 09.06.2014  & 3	&			&	1	&  4 \\
        \#2 feedback session
        				& 12.06.2014 &	1	&  1		&	1	& 3 \\
        
%template row		- 	& xx.04.2014 & x 	& x 	& x 	& x \\ 

  \end{tabular}

\end{threeparttable}

\end{table}

\begin{table}[H]
\renewcommand{\arraystretch}{1.6}
\begin{threeparttable}

  \caption{Time durations of phase: Final paper}
  
  \begin{tabular}{p{3cm} c|c c c| c}
      \textbf{Task} & \textbf{Date}& \textbf{Clear} & \textbf{Gerlach} & \textbf{Popov} & \textbf{Total hours} \\
	%					- 	& date		 & Will & Andr.	& Iar. 	& Total \\ 
        Reading Digital Democracy
                        & 13.06.2014  &		 &	2		&		&  2 \\
        Case stady Change.org
        				& 16.06.2014  &		&			&	5	&	5 \\
        Case stady Change.org
        				& 17.06.2014  &		&			&	1	&	5 \\                     		Reading Environmental
        				& 18.06.2014  &		&	2		&		&	2 \\
        Reading Social Change
        				& 21.06.2014  &		&   2		&		&	2 \\

		Editing
        				& 14.06.2014  &	 1 &	2		&		&  3 \\
         ...2.1.
                        &  20.06.2014 &		&	4		&		&  4 \\
         ...2.2.
                        &  22.06.2014 &		&	4		&		&	4 \\
         ...2.3.
        				& 24.06.2014 &		&	3		&		&	3 \\
        				& 25.06.2014 &      &   2       &       &   2
\\
        \#3 feedback session
        				&	17.06.2014	& 1	& 	1			&		&	2 \\
Research method reading (case studies, interviews)
                        & 18.06.2014 & 		2 	&		 & 		& 	2 \\
Ethics research (Stoicism, Consequentalism) 
                        &  18.06.2014 & 	3 	& 		&		 & 3 \\
Ethics research (Utilitarianism, Deontology, Postmodern ethics)
                        & 19.06.2014 & 3 & & & 3 \\
Update document to reflect feedback about research methods, phase 1
                        & 19.06.2014 & 		2 	& 		& 		& 2 \\
Legal research (Validity of petitions, Germany, USA)
                        & 20.06.2014 & 4 		& 		& 		& 4 \\
Case study research (GTA V, Pardon Snowden)
                        & 20.06.2014 & 3 & & & 3 \\

%template row		- 	& xx.04.2014 & x 	& x 	& x 	& x \\ 
\end{tabular}
\end{threeparttable}
\end{table}

\begin{table}[H]

\begin{threeparttable}

  \caption{Time durations of phase: Final paper (continue)}
  
  \begin{tabular}{p{3cm} c|c c c| c}
      \textbf{Task} & \textbf{Date}& \textbf{Clear} & \textbf{Gerlach} & \textbf{Popov} & \textbf{Total hours} \\
	%					- 	& date		 & Will & Andr.	& Iar. 	& Total \\ 

Refresh on Sociological Methods and project planning (chapter 3, Macionis \& Plummer)
                        & 20.06.2014 & 2  & & & 2 \\
		Discussion (measuring connection to victory)
        				& 21.06.2014  &	 1	&		1	&	1	&	3 \\                     
Approach for measuring connection
        				& 21.06.2014  &		&			&	2	&	2 \\                             Approach for measuring connection
        				& 22.06.2014  &		&			&	1	&	1 \\    
        Reading related literature
        				& 22.06.2014  &		&			&	3	&	3 \\    
 Approach for measuring connection
        				& 23.06.2014  &		&			&	2	&	2 \\                             
 Editing 3.2
        				& 24.06.2014  &		&			&	4	&	4 \\
        				& 25.06.2014  &	5	&			&	4	&	9 \\
        				& 26.06.2014  &	 2	&			&	5	&	7 \\
        				& 27.06.2014  &	2	&	2		&	6	&	10 \\                                             
Last Feedback Session   & 27.06.2014  &   1 &    2 &   6   &   9 \\
        				& 28.06.2014  &	6	&	4		&	6	&	16 \\  
        				
  Review                & 29.06.2014  &   7  &   7       &   7    &   21 \\



\end{tabular}

\end{threeparttable}


\end{table}

\begin{table}[H]

\begin{threeparttable}

  \caption{Time durations in total}
  
  \begin{tabular}{p{3cm} p{1.7cm}|c c c| c}
      \textbf{Task} & \textbf{Date}& \textbf{Clear} & \textbf{Gerlach} & \textbf{Popov} & \textbf{Total hours} \\
	%					- 	& date		 & Will & Andr.	& Iar. 	& Total \\ 
    
    Workload            & 15.03.2014-29.06.2014    & 101.5 & 101.5 & 101.5 & 304.5 \\
    
\end{tabular}

\end{threeparttable}


\end{table}

% section proj_plan (end)